\chapter{Black hole formation 1: dust collapse}
\label{s:lem}

\minitoc

\section{Introduction}

After having investigated black holes in equilibrium, in the form of
Schwarzschild and Kerr solutions, we turn to dynamical black hole,
more specifically to the standard process of
black hole formation: \emph{gravitational collapse}\index{gravitational!collapse}.
To deal with analytical solutions, we simplify the problem as much as
possible. First we assume spherical symmetry, which is quite natural
as a first approximation for modelling the gravitational collapse
of a stellar core or a gas cloud. A drawback is this forbids the
study of gravitational waves\index{gravitational!waves}, since by
virtue of Birkhoff theorem\index{Birkhoff!theorem} the exterior
of any spherically symmetric collapsing object is a piece of Schwarzschild
spacetime, i.e. does not contain any gravitational radiation.
The second major approximation is to consider \emph{pressureless matter},
commonly referred to as \emph{dust}\index{dust}.
An alternative, certainly more academic, is to consider the collapse of shell of pure electromagnetic radiation; this will be considered in Chap.~\ref{s:vai}.

\section{Lemaître-Tolman equations}

\subsection{From the Einstein equation to the Lemaître-Tolman system}

As mentionned in the Introduction, we shall restrict ourselves to
spherically symmetric\footnote{See Sec.~\ref{s:sch:static_spher} for a precise
definition of \emph{spherically symmetric}.} spacetimes, and for concretness, to 4-dimensional ones. The most general spherically symmetric 4-dimensional spacetime $(\M,\w{g})$
can be described in terms of coordinates $x^\alpha=(\tau,\chi,\th,\ph)$ such that
the components of the metric tensor are written
\be
    g_{\mu\nu}\, \D x^\mu \, \D x^\nu = - \D\tau^2 + a(\tau,\chi)^2 \D\chi^2
        + r(\tau,\chi)^2 \left( \D\th^2 + \sin^2\th\, \D\ph^2 \right)  ,
\ee
where $a(\tau,\chi)$ and $r(\tau,\chi)$ are generic (non-vanishing) functions.
These coordinates are called \defin{Lemaître synchronous coordinates}\index{Lemaitre@Lemaître!synchronous coordinates}\index{synchronous!coordinates}, the qualifier
\defin{synchronous} meaning that $\tau$ is the proper time of a observer staying
at fixed value of the spatial coordinates $(\chi,\th,\ph)$.

For simplification, we consider only a pressureless matter, in the form of a perfect
fluid of 4-velocity $\w{u}$ with zero pressure. The matter energy-momentum tensor is then
\be
    \w{T} = \rho \w{u} \otimes \w{u} ,
\ee
where the scalar field $\rho$ can be interpreted as the fluid energy density
measured in the fluid frame.
Inside the matter, we link the coordinates $(\tau,\chi,\th,\ph)$ to the fluid by demanding
that they are \defin{comoving}\index{comoving!coordinates} with the fluid, i.e. that a fluid particle
stays at fixed values of $(\chi,\th,\ph)$.
Because the 4-velocity obeys $u^\alpha = \D x^\alpha/\D \tau_{\rm fl}$, where
$\tau_{\rm fl}$ is the fluid proper time [cf. Eq.~(\ref{e:fra:def_u})], this
amounts to set $u^\chi=u^\th = u^\ph = 0$, i.e. to have
\be
    \w{u} = \wpar_\tau .
\ee
A priori, one should have only $\w{u} =  u^\tau\wpar_\tau$, but the
synchronous coordinate condition $g_{\tau\tau} = -1$ along with the
normalization $\w{g}(\w{u},\w{u})=-1$ implies $u^\tau=1$. Since $u^\tau = \D\tau / \D\tau_{\rm fl}$, we get $\tau = \tau_{\rm fl}$ (up to some additive constant), which provides the physical
interpretation of Lemaître coordinate $\tau$ as the fluid proper time.

The expression of the Einstein equation (\ref{e:fra:Einstein_eq})  in terms of Lemaître synchronous coordinates
is performed in Sec.~\ref{s:sam:Lemaitre-Tolman}.

\section{Oppenheimer-Snyder solution}


