\chapter{Black hole formation 1: dust collapse}
\label{s:lem}

\minitoc

\section{Introduction}

After having investigated black holes in equilibrium, in the form of
Schwarzschild and Kerr solutions, we turn to dynamical black hole,
more specifically to the standard process of
black hole formation: \emph{gravitational collapse}\index{gravitational!collapse}.
To deal with analytical solutions, we simplify the problem as much as
possible. First we assume spherical symmetry, which is quite natural
as a first approximation for modelling the gravitational collapse
of a stellar core or a gas cloud. A drawback is this forbids the
study of gravitational waves\index{gravitational!waves}, since by
virtue of Birkhoff theorem\index{Birkhoff!theorem} the exterior
of any spherically symmetric collapsing object is a piece of Schwarzschild
spacetime, i.e. does not contain any gravitational radiation.
The second major approximation is to consider \emph{pressureless matter},
commonly referred to as \emph{dust}\index{dust}.
An alternative, certainly more academic, is to consider the collapse of shell of pure electromagnetic radiation; this will be considered in Chap.~\ref{s:vai}.

\section{Lemaître-Tolman equations}

\subsection{Hypotheses}

As mentionned in the Introduction, we shall restrict ourselves to
spherically symmetric\footnote{See Sec.~\ref{s:sch:static_spher} for a precise
definition of \emph{spherically symmetric}.} spacetimes, and for concretness, to 4-dimensional ones. The most general spherically symmetric 4-dimensional spacetime $(\M,\w{g})$
can be described in terms of coordinates $x^\alpha=(\tau,\chi,\th,\ph)$ such that
the components of the metric tensor are written
\be \label{e:lem:metric_sync_coord}
    g_{\mu\nu}\, \D x^\mu \, \D x^\nu = - \D\tau^2 + a(\tau,\chi)^2 \D\chi^2
        + r(\tau,\chi)^2 \left( \D\th^2 + \sin^2\th\, \D\ph^2 \right)  ,
\ee
where $a(\tau,\chi)$ and $r(\tau,\chi)$ are generic positive functions.
These coordinates are called \defin{Lemaître synchronous coordinates}\index{Lemaitre@Lemaître!synchronous coordinates}\index{synchronous!coordinates}, the qualifier
\defin{synchronous} meaning that $\tau$ is the proper time of a observer staying
at fixed value of the spatial coordinates $(\chi,\th,\ph)$.
Note that the function $r(\tau,\chi)$ gives the \emph{areal radius}\index{areal!radius}
of the 2-spheres
defined by $(\tau,\chi) = \mathrm{const}$, which are the orbits of the $\mathrm{SO}(3)$
group action (cf. Sec.~\ref{s:sch:static_spher}), i.e. the metric area of these
2-spheres is $4\pi r(\tau,\chi)^2$.

For simplification, we consider only a pressureless matter, in the form of a perfect
fluid of 4-velocity $\w{u}$ with zero pressure. The matter energy-momentum tensor is then
\be \label{e:lem:T_pressureless}
    \w{T} = \rho \w{u} \otimes \w{u} ,
\ee
where the scalar field $\rho$ can be interpreted as the fluid energy density
measured in the fluid frame. Let us recall that the energy-mometum tensor of a generic
perfect fluid is $\w{T} = (\rho + p)\w{u} \otimes \w{u} + p \w{g}$, where $p$
is the fluid pressure. The expression (\ref{e:lem:T_pressureless}) corresponds thus
to the special case $p=0$.
Inside the matter, we link the coordinates $(\tau,\chi,\th,\ph)$ to the fluid by demanding
that they are \defin{comoving}\index{comoving!coordinates} with the fluid, i.e. that a fluid particle
stays at fixed values of $(\chi,\th,\ph)$.
Because the 4-velocity obeys $u^\alpha = \D x^\alpha/\D \tau_{\rm fl}$, where
$\tau_{\rm fl}$ is the fluid proper time [cf. Eq.~(\ref{e:fra:def_u})], this
amounts to set $u^\chi=u^\th = u^\ph = 0$, i.e. to have
\be \label{e:lem:u_par_tau}
    \w{u} = \wpar_\tau .
\ee
A priori, one should have only $\w{u} =  u^\tau\wpar_\tau$, but the
synchronous coordinate condition $g_{\tau\tau} = -1$ along with the
normalization $\w{g}(\w{u},\w{u})=-1$ implies $u^\tau=1$. Since $u^\tau = \D\tau / \D\tau_{\rm fl}$, we get $\tau = \tau_{\rm fl}$ (up to some additive constant), which provides the physical
interpretation of Lemaître coordinate $\tau$ as the fluid proper time.

\subsection{Geodesic matter flow}

The equation of energy-momentum conservation $\wnab\cdot\vw{T} = 0$
[Eq.~(\ref{e:fra:divT})], which
follows from the Einstein equation (\ref{e:fra:Einstein_eq}) and the contracted
Bianchi identity (\ref{e:bas:Bianchi_contr}) (cf. Sec.~\ref{s:fra:Einstein_eq}),
implies that
\begin{greybox}
The worldlines of the fluid particles obeying the pressureless matter
model (\ref{e:lem:T_pressureless}) are timelike geodesics of
spacetime $(\M,\w{g})$.
\end{greybox}
\begin{proof}
If we plug the energy-momentum tensor (\ref{e:lem:T_pressureless}) in the
energy-momentum conservation law (\ref{e:fra:divT}), we obtain
\[
    \nabla_\mu (\rho u^\mu u^\alpha )  = 0 ,
\]
i.e.
\be \label{e:lem:divT_pressureless}
    \nabla_\mu (\rho u^\mu) u^\alpha + \rho u^\mu \nabla_\mu u^\alpha = 0 .
\ee
Now the two terms in the left-hand side of this equation are orthogonal
to each other, as an immediate consequence of the normalization of the
4-velocity $\w{u}$ [Eq.~(\ref{e:fra:u_unit})]:
$\w{u}\cdot \wnab_{\w{u}}\w{u} = 0$. In particular, $\w{u}$
is a timelike vector, while the 4-acceleration $\wnab_{\w{u}}\w{u}$
is a spacelike one. The only way for Eq.~(\ref{e:lem:divT_pressureless})
to hold is thus that each term
in the left-hand side vanishes separately:
\[
    \nabla_\mu (\rho u^\mu) = 0 \qquad\mbox{and}\qquad u^\mu \nabla_\mu u^\alpha = 0 .
\]
The second equation above is nothing but the geodesic equation for the
field lines of $\w{u}$, i.e. the fluid wordlines.
\end{proof}
Each fluid particle is thus in free-fall and moves independently of its
neighbours, which is not surprising since the pressure is zero.
This justify the term \defin{dust}\index{dust} given to the matter
model (\ref{e:lem:T_pressureless}).

\subsection{From the Einstein equation to the Lemaître-Tolman system}

Let us write the Einstein equation (\ref{e:fra:Einstein_eq})
in terms of Lemaître synchronous coordinates $(\tau,\chi,\th,\ph)$
and with the energy-momentum tensor (\ref{e:lem:T_pressureless})-(\ref{e:lem:u_par_tau})
in its right-hand side.
As detailed in Sec.~\ref{s:sam:Lemaitre-Tolman}, the $\tau\chi$ component yields
\be \label{e:lem:a_f_dr}
    a(\tau,\chi) = \frac{1}{f(\chi)} \der{r}{\chi} ,
\ee
where $f(\chi)$ is an arbitrary function of $\chi$.
Accordingly, we may rewrite the metric components (\ref{e:lem:metric_sync_coord})
\be \label{e:lem:metric_Lemaitre}
    \encadre{ g_{\mu\nu}\, \D x^\mu \, \D x^\nu = - \D\tau^2
        + \frac{1}{f(\chi)^2} \left( \der{r}{\chi} \right)^2 \D\chi^2
        + r(\tau,\chi)^2 \left( \D\th^2 + \sin^2\th\, \D\ph^2 \right) } .
\ee

Taking into account (\ref{e:lem:a_f_dr}), the $\chi\chi$ and $\tau\tau$ components of the Einstein equation
yield respectively to (cf. Sec.~\ref{s:sam:Lemaitre-Tolman})
\begin{subequations}\label{e:lem:LTeqs}
\begin{align}
 & \encadre{\left( \der{r}{\tau} \right) ^2 = f(\chi)^2 - 1 + \frac{2m(\chi)}{r(\tau,\chi)}
   + \frac{\Lambda}{3} r(\tau,\chi)^2 } \label{e:lem:LTeqs1} \\
 & \encadre{\frac{\D m}{\D\chi} = 4\pi r(\tau,\chi)^2 \rho(\tau,\chi) \der{r}{\chi} } ,
    \label{e:lem:LTeqs2}
 \end{align}
\end{subequations}
where $m(\chi)$ is another arbitrary function of $\chi$.
There is no other independent component of Einstein equation.
Equations~(\ref{e:lem:LTeqs}) constitute the
\defin{Lemaître-Tolman system}\index{Lemaitre-Tolman system@Lemaître-Tolman system}.

The function $m(\chi)$ is known in the literature as the \defin{Misner-Sharp mass}\index{Misner-Sharp!mass} or \defin{Misner-Sharp energy}\index{Misner-Sharp!energy}, in reference
of a study by Misner and Sharp in 1964 \cite{MisneS64}, despite it has been introduced
more than 30 years earlier by Lemaître \cite{Lemai32}. This quantity is
invariantly defined for any spherically symmetric spacetime from the areal radius $r$:
\be \label{e:lem:def_Misner_Sharp}
    m  := \frac{r}{2} \left( 1 - \nabla_\mu r \nabla^\mu r  - \frac{\Lambda}{3} r^2\right) .
\ee
It is easy to check that the above relation holds in the present case:
we have, thanks to (\ref{e:lem:metric_sync_coord}),
\[
    \nabla_\mu r \nabla^\mu r = g^{\mu\nu} \der{r}{x^\mu} \der{r}{x^\nu}
        = g^{\tau\tau} \left( \der{r}{\tau} \right)^2
        + g^{\chi\chi} \left( \der{r}{\chi} \right)^2
        = - \left( \der{r}{\tau} \right)^2 + \frac{1}{a(\tau,\chi)^2} \left( \der{r}{\chi} \right)^2
\]
Using Eq.~(\ref{e:lem:a_f_dr}), this expression reduces to
\[
    \nabla_\mu r \nabla^\mu r = - \left( \der{r}{\tau} \right)^2 + f(\chi)^2 .
\]
In view of the Lemaître-Tolman equation (\ref{e:lem:LTeqs1}), we conclude that
(\ref{e:lem:def_Misner_Sharp}) holds.

\subsection{Solutions for a vanishing cosmological constant}

In the rest of this chapter, we assume $\Lambda=0$, since we are mainly
interested in gravitational collapse in asymptotically flat spacetimes.
The Lemaître-Tolman equation (\ref{e:lem:LTeqs1}) can be then rewritten
as
\be \label{e:lem:1dim_mechanical}
    \frac{1}{2} \dot{r}^2 - \frac{m(\chi)}{r} = E(\chi) ,
\ee
where $\dot{r} := \partial r /\partial\tau$ and
\be \label{e:lem:E_f_chi}
    E(\chi) := \frac{f(\chi)^2-1}{2} .
\ee
For a fixed value of $\chi$, we recognize
the equation ruling the 1-dimensional non-relativistic motion of a
particle in a Newtonian
potential $V=-m/r$; $E(\chi)$ is then nothing but the total
mechanical energy of the particle per unit mass.
As it is well known, the solution of (\ref{e:lem:1dim_mechanical})
depends on the sign of $E(\chi)$:
\begin{itemize}
\item if $E(\chi)>0$, the solution is given in parametrized form (parameter $\eta$) by
\be \label{e:lem:sol_E_pos}
    \left\{ \begin{array}{l}
    \displaystyle\tau = \frac{m(\chi)}{(2E(\chi))^{3/2}} \left( \sinh\eta - \eta \right)
        + \tau_0(\chi) \\[2ex]
    \displaystyle r(\tau,\chi) = \frac{m(\chi)}{2E(\chi)} \left( \cosh\eta - 1 \right)
    \end{array} \right.
\ee
\item if $E(\chi)=0$, the solution is
\be \label{e:lem:sol_E_zero}
    r(\tau,\chi) =  \left( \frac{3}{2} \sqrt{2m(\chi)} \left| \tau -\tau_0(\chi) \right| \right) ^{2/3}
\ee
\item if $E(\chi)<0$, the solution is given in parametrized form (parameter $\eta$) by
\be \label{e:lem:sol_E_neg}
    \left\{ \begin{array}{l}
    \displaystyle\tau =  \frac{m(\chi)}{|2E(\chi)| ^{3/2}} \left( \eta + \sin\eta \right)
    + \tau_0(\chi)  \\[2ex]
    \displaystyle r(\tau,\chi) = \frac{m(\chi)}{|2E(\chi)|} \left( 1 + \cos\eta \right)
    \end{array} \right.
\ee
\end{itemize}
In the above formulas, $\tau_0(\chi)$ is an arbitrary function of $\chi$.
For $E>0$ and $E=0$, it sets the value of $\tau$ for which $r=0$, while
for $E<0$, it sets the value of $\tau$ for which $r$ takes its maximal value
($m/|E|$).

\noindent\emph{Exercise:} prove that each of formulas (\ref{e:lem:sol_E_pos})-(\ref{e:lem:sol_E_neg}) provides
a solution of Eq.~(\ref{e:lem:1dim_mechanical}).

The procedure to get a full solution is (i) choose the functions
$f(\chi)$, $m(\chi)$ and $\tau_0(\chi)$;
(ii) evaluate $E(\chi)$ via (\ref{e:lem:E_f_chi}); (iii) depending of
on the value of $E(\chi)$, use (\ref{e:lem:sol_E_pos}), (\ref{e:lem:sol_E_zero})
or (\ref{e:lem:sol_E_neg}) to get the solution for $r(\tau,\chi)$;
(iv) plug this solution into the remaining Lemaître-Tolman equation,
Eq.~(\ref{e:lem:LTeqs2}), to get $\rho(\tau,\chi)$ and into
(\ref{e:lem:metric_Lemaitre}) to get the metric tensor.

\subsection{Schwarzschild solution in Lemaître coordinates}

One can recover Schwarzschild solution from the above setting by
considering the vacuum case, i.e. $\rho=0$. Then
Eq.~(\ref{e:lem:LTeqs2}) imposes $m(\chi)$ to be a constant, which
we shall denote simply by $m$. Regarding the function $f(\chi)$, let us
choose for simplicity $f(\chi)=1$. Then $E(\chi)=0$ and $r(\tau,\chi)$
is given by Eq.~(\ref{e:lem:sol_E_zero}). Since $m$ is constant, we
cannot choose $\tau_0(\chi)$ to be a constant, otherwise
$\partial r/\partial \chi$ would be zero and the metric
(\ref{e:lem:metric_Lemaitre}) would be degenerate.
The simplest non-constant choice is
$\tau_0(\chi) = \chi$. To summarize, the
three functions of $\chi$ determining the solution are set to
\be
    m(\chi)=m=\mathrm{const},\qquad f(\chi) = 1 \qquad\mbox{and}\qquad \tau_0(\chi) = \chi .
\ee
Equation~(\ref{e:lem:sol_E_zero}), with the above values for $m(\chi)$ and $\tau_0(\chi)$
and assuming $\tau\geq \chi$,
yields
\be \label{e:lem:r_tau_chi_Schwarz}
    r(\tau,\chi) =  \left( \frac{3}{2} \sqrt{2m} ( \tau -\chi ) \right) ^{2/3} .
\ee
Then
\[
    \der{r}{\chi} = - \sqrt{2m} \left( \frac{3}{2} \sqrt{2m} ( \tau -\chi ) \right) ^{-1/3}
        = - \sqrt{\frac{2m}{r}} ,
\]
so that Eq.~(\ref{e:lem:metric_Lemaitre}) becomes
\be \label{e:lem:Sch_met_Lem}
    \encadre{  g_{\mu\nu}\, \D x^\mu \, \D x^\nu = - \D\tau^2
        + \frac{2m}{r} \D\chi^2
        + r^2 \left( \D\th^2 + \sin^2\th\, \D\ph^2 \right)  } .
\ee
In this expression, $r$ is the function of $(\tau,\chi)$ given
by (\ref{e:lem:r_tau_chi_Schwarz}).

The metric (\ref{e:lem:Sch_met_Lem}) is actually the
Schwarzschild metric of mass parameter $m$. To prove it, let us first
move to coordinates $(\tau,r,\th,\ph)$, the relation to Lemaître coordinates
${x'}^\alpha=(\tau,\chi,\th,\ph)$ is obtained by differentiating (\ref{e:lem:r_tau_chi_Schwarz}):
\[
    \D r =  \sqrt{2m} \left( \frac{3}{2} \sqrt{2m} ( \tau -\chi ) \right) ^{-1/3}
       (\D\tau - \D\chi) = \sqrt{\frac{2m}{r}} (\D\tau - \D\chi) ,
\]
from which
\[
  \sqrt{\frac{2m}{r}}  \D\chi = \sqrt{\frac{2m}{r}}  \D\tau - \D r
\quad \Longrightarrow \quad \frac{2m}{r} \D\chi^2 = \frac{2m}{r} \D\tau^2
    - 2 \sqrt{\frac{2m}{r}}  \D\tau \, \D r + \D r^2 .
\]
Substituting this relation in Eq.~(\ref{e:lem:Sch_met_Lem}) yields
\bea
    {g'}_{\mu\nu}\, \D {x'}^\mu \, \D {x'}^\nu & = &
     - \left( 1 - \frac{2m}{r} \right)
        \D\tau^2 - 2 \sqrt{\frac{2m}{r}}  \D\tau \, \D r + \D r^2
        + r^2 \left( \D\th^2 + \sin^2\th\, \D\ph^2 \right) \nonumber \\
    & = & - \left( 1 - \frac{2m}{r} \right) \left( \D\tau^2 + 2
        \frac{\sqrt{\frac{2m}{r}}}{1 - \frac{2m}{r}} \, \D\tau \, \D r \right)
            + \D r^2 \nonumber \\
        & & + r^2 \left( \D\th^2 + \sin^2\th\, \D\ph^2 \right) \nonumber \\
    & = & - \left( 1 - \frac{2m}{r} \right) \left( \D\tau +
        \frac{\sqrt{\frac{2m}{r}}}{1 - \frac{2m}{r}} \, \D r \right) ^2
            + \frac{\D r^2}{1- \frac{2m}{r}}
            + r^2 \left( \D\th^2 + \sin^2\th\, \D\ph^2 \right) . \label{e:lem:met_tau_r}
\eea
If we introduce, instead of $\tau$, a coordinate $t$ such that
\be \label{e:lem:dt_dtau_dr}
    \D t = \D\tau +
        \frac{\sqrt{\frac{2m}{r}}}{1 - \frac{2m}{r}} \, \D r ,
\ee
Eq.~(\ref{e:lem:met_tau_r}) yields immediately to the familiar expression
of Schwarzschild metric in Schwarzschild-Droste coordinates $(t,r,\th,\ph)$
[Eq.~(\ref{e:sch:Schwarz_metric_SD})].
The relation between Lemaître coordinates and
and Schwarzschild-Droste ones can be made explicit by integrating
Eq.~(\ref{e:lem:dt_dtau_dr}); one gets, up to some constant,
\be
    \tau = t - 2\sqrt{2m r} - 2 m \ln \left(
        \frac{\sqrt{r} - \sqrt{2m}}{\sqrt{r} + \sqrt{2m}} \right) .
\ee
The expression of $\chi$ in terms of $(t,r)$ is then deduced from
Eq.~(\ref{e:lem:r_tau_chi_Schwarz}):
\be
    \chi = t - 2\sqrt{2m r} \left( 1 + \frac{r}{6m} \right) - 2 m \ln \left(
        \frac{\sqrt{r} - \sqrt{2m}}{\sqrt{r} + \sqrt{2m}} \right) .
\ee

\section{Oppenheimer-Snyder solution}





















