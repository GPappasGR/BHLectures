\chapter{SageMath computations} \label{s:sam}

\minitoc

\section{Introduction}


\textsf{SageMath} (\url{http://sagemath.org/}) is a modern free,
open-source mathematics software system, which is
based on the Python programming language. It makes use of over 90 open-source packages,
among which are \textsf{Maxima} and \textsf{Pynac} (symbolic calculations),
\textsf{GAP} (group theory),
\textsf{PARI/GP} (number theory), \textsf{Singular} (polynomial computations),
and \textsf{matplotlib} (high quality 2D figures).
\textsf{SageMath} provides a uniform Python interface to all these packages; however,
\textsf{SageMath} is much more than a mere interface: it contains a large and increasing part of
original code (more than 750,000 lines of Python and Cython, involving 5344 classes).
\textsf{SageMath} was created in 2005 by W. Stein \cite{SteinJ05} and since
then its development has been sustained by more than a hundred researchers
(mostly mathematicians). Very good introductory textbooks about \textsf{SageMath} are
\cite{JoyneS14,Zimme13,Bard15}.

The \textsf{SageManifolds} extension to
\textsf{SageMath} provides capability for differential geometry and tensor calculus
(\url{http://sagemanifolds.obspm.fr/}), which we are using here to perform
some computations related to these lectures.

There are basically two ways to use \textsf{SageMath} + \textsf{SageManifolds}:
\begin{itemize}
\item Install it on your computer, by downloading the sources or a binary version
from \url{http://sagemath.org/}; \textsf{SageManifolds} is progressively integrated
into \textsf{SageMath}, but this step-by-step process is not finished yet. So
one has to install \textsf{SageManifolds} in addition to \textsf{SageMath}, by
following the instructions\footnote{On a Linux system, this reduces to simply type
two commands in a terminal:\\
\texttt{wget -N http://sagemanifolds.obspm.fr/spkg/sm-install.sh}\\
\texttt{bash sm-install.sh}}
at \\ \url{http://sagemanifolds.obspm.fr/download.html}.
\item Use it online at the \textsf{SageMathCloud} (\url{https://cloud.sagemath.com/});
since \textsf{SageManifolds} is installed on \textsf{SageMathCloud}, you can
use it directly.
\end{itemize}

\section{SageManifolds worksheets}

The SageManifolds worksheets accompanying these lecture notes are available
at

\centerline{
\url{http://luth.obspm.fr/~luthier/gourgoulhon/bh16/sage.html}}

