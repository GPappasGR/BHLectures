\chapter{The concept of black hole}

\minitoc

\section{Introduction}

\section{General framework}

\subsection{Spacetime}

In these lectures we consider a $n$-dimensional \defin{spacetime}\index{spacetime},
i.e. a pair $(\M, \w{g})$, where $\M$ is a $n$-dimensional smooth manifold, with $n\geq 2$, and $\w{g}$ is a
Lorentzian metric on $\M$ (cf. Sec.~\ref{s:bas:pRiemManif} in
Appendix~\ref{s:bas}). In many parts, the dimension $n$ will be set to 4
--- the standard spacetime value --- but we shall also consider spacetimes with
$n>4$, especially in Chap.~??.

The precise definition and basic properties of a \emph{smooth manifold} are recalled
in Appendix~\ref{s:bas}. Here let us simply say that, in loose terms,
a \defin{manifold}\index{manifold} $\M$ of dimension $n$ is a ``space'' that \emph{locally} resembles $\R^n$,
i.e. can be described by a $n$-tuple of coordinates $(x^1,\ldots,x^n)$. Globally,
$\M$ can be very different from $\R^n$, in particular regarding its topology.
The smooth structure endows the manifold
with the concept of infinitesimal vectors\index{vector!infinitesimal --} $\D\w{x}$, which connect infinitely
close pair of points of $\M$. However, for finitely separated points, there is no
longer the concept of vector connecting them (contrary for instance to points in $\R^n$).
In other words, vectors on $\M$ do not live in the manifold but in the
tangent spaces $T_p\M$, which are defined at each point $p\in\M$. Each  $T_p\M$
is a $n$-dimensional vector space, which is generated for instance by the infinitesimal displacement
vectors along the $n$ coordinate lines of some coordinate system.

We shall specify further the spacetime structure later on, in particular its
asymptotics and its orientability.

\subsection{Worldlines}

In relativity, a particle is described by its spacetime extent, which is a smooth curve,
$\mathscr{L}$ say, and not a point. This curve is called the particle's
\defin{worldline}\index{worldline} and might be thought of as the set of
the ``successive positions'' occupied by the particle as ``time evolves''.
The dynamics of a simple particle (i.e. a particle without internal structure nor
spin) is entirely described by its
\defin{4-momentum}\index{4-momentum} or \defin{energy-momentum vector}\index{energy-momentum!vector}\footnote{When $n\not=4$, \emph{energy-momentum vector} is definitely a better name
than \emph{4-momentum}!}, which is a vector field $\w{p}$ defined along $\mathscr{L}$
and tangent to $\mathscr{L}$ at each point (cf. Fig.~??).

One distinguishes two types of particles:
\begin{itemize}
\item the \defin{massive particles}\index{massive!particle}\index{particle!massive --}, for which
$\mathscr{L}$ is a timelike curve, or equivalently, for which
$\w{p}$ is a timelike vector:
\be
    \w{g}(\w{p},\w{p}) = \w{p}\cdot\w{p} < 0 ;
\ee
one says then that $\mathscr{L}$ is a timelike curve
\item the \defin{massless particles}\index{massless!particle}\index{particle!massless --},
such as the photon,
for which $\mathscr{L}$ is a null curve, or equivalently, for which  $\w{p}$ is a null vector:
\be
    \w{g}(\w{p},\w{p}) = \w{p}\cdot\w{p} = 0 .
\ee
\end{itemize}
In both cases, the \defin{mass} of the particle is defined by
\be
   m = \sqrt{- \w{p}\cdot\w{p}} .
\ee
Of course, for a massless particle, we get $m=0$.

If the particle feels only gravitation, i.e. if no non-gravitational force
is exerted on it, the energy-momentum vector must be a
\defin{geodesic vector}\index{geodesic!vector}, i.e. it obeys
\be \label{e:def:p_geodesic}
    \encadre{\wnab_{\w{p}}\,  \w{p} = 0 } ,
\ee
or, in terms of components
\be
    p^\mu \nabla_\mu p^\alpha = 0 .
\ee
This implies that the worldline $\mathscr{L}$ must be a
\defin{geodesic}\index{geodesic} of the spacetime $(\M,\w{g})$.
\begin{remark}
The reverse is not true, i.e. having $\mathscr{L}$ geodesic and $\w{p}$
tangent to $\mathscr{L}$ does not imply (\ref{e:def:p_geodesic}), but the
weaker condition $\wnab_{\w{p}}\,  \w{p} = \alpha \, \w{p}$, with $\alpha$
a scalar field along $\mathscr{L}$.
\end{remark}
For massive particles, Eq.~(\ref{e:def:p_geodesic}) can be derived from
a least-action principle

\subsubsection{Massive particles}

For a massive particle, the constraint of having $\mathscr{L}$ timelike
has a simple geometrical meaning: the worldline $\mathscr{L}$ must
always lie inside the light cones of events along $\mathscr{L}$ (cf. Fig.~??).

\subsection{Einstein equation}

Saying that gravitation in the spacetime $(\M, \w{g})$ is ruled by
\defin{general relativity}\index{general!relativity} amounts
to demanding that the metric $\w{g}$ obeys \defin{Einstein equation}\index{Einstein!equation}:
\be \label{e:bas:Einstein_eq}
    \encadre{ \w{R} - \frac{1}{2}\, R\, \w{g} + \Lambda\, \w{g} = 8\pi \w{T} },
\ee
where $\w{R}$ is the Ricci tensor of $\w{g}$, $R$ the Ricci scalar of $\w{g}$
(cf. Sec.~\ref{s:bas:Ricci_tensor} in Appendix~\ref{s:bas}), $\Lambda$ some
constant, called the \defin{cosmological constant}\index{cosmological!constant}
and $\w{T}$ is the energy-momentum tensor\index{energy-momentum tensor} of
matter and non-gravitational fields.
\begin{remark}
The dimension $n$ of the spacetime does not appear in the Einstein equation
(\ref{e:bas:Einstein_eq}).
\end{remark}



\section{A first definition of black holes}

\subsection{Black holes and null hypersurfaces}

A naive definition of a black hole, involving only words, could be
\begin{quote}
A \defin{black hole}\index{black!hole} is a localized region of spacetime
from which neither massive particles nor light may escape.
\end{quote}
There are essentially two features in this definition: localization
and inescapability. Let us for a moment focus on the latter.
It implies the existence of a \emph{boundary}, which no
(massive or not) particle emitted in the black hole region can cross.
This boundary is called the
\defin{event horizon}\index{event!horizon}\index{horizon!event --} and
quite often referred to simply as the \defin{horizon}.
It is a \defin{one-way membrane}\index{one-way membrane}\index{membrane!one-way --},
in the sense that it can be crossed from the black hole ``exterior'' towards
the black hole region, but not in the reverse way. The one-way membrane must
a hypersurface of the spacetime manifold $\M$, for it has to divide $\M$ in two regions:
the interior (the black hole itself) and the exterior region.
Let us recall that a hypersurface is a submanifold of $\M$ of codimension 1
(cf. Sec.~\ref{s:bas:embed} in Appendix~\ref{s:bas}).

To discuss further which hypersurface could act as a black hole boundary,
one should recall that, on a Lorentzian manifold $(\M,\w{g})$, there are
three types of hypersurfaces. The classification
depends on the type of metric induced by $\w{g}$ on the
hypersurface, $\Sigma$ say, the
\defin{induced metric}\index{induced!metric}\index{metric!induced --} being
nothing but the restriction $\left.\w{g}\right| _{\Sigma}$ of $\w{g}$
to vector fields tangent to $\Sigma$.
A hypersurface $\Sigma$ is said to be
\begin{itemize}
\item \defin{spacelike} iff $\left.\w{g}\right| _{\Sigma}$ is positive definite,
i.e. iff $\mathrm{sign} \left.\w{g}\right| _{\Sigma} = (+,+,+)$,
i.e. iff $(\Sigma,  \left.\w{g}\right| _{\Sigma})$ is a Riemannian manifold;
\item \defin{timelike} iff $\left.\w{g}\right| _{\Sigma}$ is a Lorentzian metric,
i.e. iff $\mathrm{sign} \left.\w{g}\right| _{\Sigma} = (-,+,+)$,
i.e. iff $(\Sigma,  \left.\w{g}\right| _{\Sigma})$ is a Lorentzian manifold;
\item \defin{null} iff $\left.\w{g}\right| _{\Sigma}$ is degenerate\footnote{
cf. Sec.~\ref{s:bas:metric} in Appendix~\ref{s:bas} for the definition of a
degenerate bilinear form; the degeneracy
implies that the bilinear form $\left.\w{g}\right| _{\Sigma}$ is not a metric on $\Sigma$.}
i.e. iff $\mathrm{sign} \left.\w{g}\right| _{\Sigma} = (0,+,+)$.
\end{itemize}
The hypersurface type can also be deduced from the normal vector
$\w{n}$ to it (cf. Sec.~\ref{s:bas:hyp_normal}):
\begin{itemize}
\item $\Sigma$ spacelike $\iff$ $\w{n}$ timelike;
\item $\Sigma$ timelike $\iff$ $\w{n}$ spacelike;
\item $\Sigma$ null $\iff$ $\w{n}$ null.
\end{itemize}
These equivalences are easily proved by considering a $\w{g}$-orthogonal basis
adapted to $\Sigma$.

\begin{remark}
Null hypersurfaces have the distinctive feature that their normals are
also tangent to them. Indeed, saying that the normal $\w{n}$ is null
is by definition equivalent to $\w{n}\cdot\w{n}=0$, which is nothing but the condition
for $\w{n}$ to be tangent to $\Sigma$.
\end{remark}

Only null and spacelike hypersurfaces are eligible as a one-way membrane
regarding timelike and null worldlines (cf. Fig.~??).
The limit case between two-way membranes (timelike hypersurfaces)
and one-way ones being null hypersurfaces, it is quite natural to select the
latter ones for the black hole boundary, rather than spacelike hypersurfaces.
Note however that in Chap.~??, we shall see that spacelike hypersurfaces,
called \emph{dynamical
horizons}\index{dynamical!horizon}\index{horizon!dynamical --}, are involved
in the quasi-local approaches to black holes.

\subsection{Geometry of null hypersurfaces}

Having decided that the black hole event horizon must be a null hypersurface,
let us examine the geometry properties of such a hypersurface. We shall
denote it by $\Hor$, for \emph{horizon}, but the results of this section
will be valid for any null hypersurface.


Locally, the hypersurface $\Hor$ can be considered as a level set, i.e. there exists
a smooth scalar field $u:\ \Hor \rightarrow \R$ such that
\be \label{e:def:Hor_u_zero}
    \forall p \in \M,\quad p\in \Hor \iff u(p) = 0 .
\ee
For example, if $\Hor$ is a light cone in Minkowsky spacetime and
$(t,r,\th,\ph)$ are standard spherical coordinates, we may choose
the retarded time coordinate $u=t-r$.

Let $\wl$ be a vector field normal to $\Hor$. Since $\Hor$ is a null hypersurface,
$\wl$ is a null vector:
\be
    \wl\cdot\wl = 0
\ee
\begin{remark}
As a consequence of the above property, there is no natural normalization
of $\wl$, contrary to the case of timelike or spacelike hypersurfaces,
where one can always choose the normal to be a unit vector
(scalar square equal to $1$ or $-1$). It follows that there is no unique choice
of $\wl$. At this stage, any rescaling $\wl \mapsto \wl' =  \lambda \wl$, with
$\lambda$ a strictly positive (or strictly negative) scalar field on $\Hor$,
yields a normal vector field $\wl'$ as valid as $\wl$.
\end{remark}
The null normal vector field $\wl$ is a priori defined on $\Hor$
only and not at points $p\not\in\Hor$.
However, it is worth to consider $\wl$ as a vector field
not confined to $\Hor$ but defined
in some open subset of $\M$ around $\Hor$.
In particular this would permit to define the spacetime covariant
derivative $\w{\nabla}\wl$, which is not possible if the
support of $\wl$ is restricted to $\Hor$.
Following Carter \cite{Carte97}, a simple way to achieve
this is to consider not only a single null hypersurface $\Hor$,
but a foliation of $\M$ (in the vicinity
of $\Hor$) by a family of null hypersurfaces, such that $\Hor$ is an
element of this family.
Without any loss of generality,
we may select the value of the scalar field $u$ to label these hypersurfaces and
denote the family by $(\Hor_u)$. The null hypersurface $\Hor$
is then nothing but the element $\Hor = \Hor_{u=0}$ of this family
[Eq.~(\ref{e:def:Hor_u_zero})].
The vector field $\wl$ can then be viewed as defined in the part of $\M$
foliated by $(\Hor_u)$, such that at each point in this region, $\wl$
is null and normal to $\Hor_u$ for some value of $u$.

Obviously the family $(\Hor_u)$ is non-unique but all geometrical
quantities that we shall introduce hereafter do not depend upon the choice
of the foliation $\Hor_u$ once they are evaluated at $\Hor$.

Since $\Hor$ is a hypersurface where $u$ is constant [Eq.~(\ref{e:def:Hor_u_zero})],
we have, by definition,
\bea
    \forall \w{v}\in T_p\M,\quad \w{v} \mbox{\ tangent to\ }\Hor & \iff  & \langle \wnab u , \w{v} \rangle = 0 \nonumber \\
    & \iff & \vw{\nabla} u \cdot \w{v} = 0 ,   \label{e:def:nab_u_normal}
\eea
where $\vw{\nabla} u$ is the gradient vector field of the scalar field $u$.
Property (\ref{e:def:nab_u_normal}) means that $\vw{\nabla} u$ is
a normal vector field to $\Hor$. By uniqueness of the normal direction, it
must then be colinear to $\wl$. Therefore, there must exist some scalar
field $\rho$ such that
\be \label{e:def:wl_rho_u}
    \encadre{\wl = e^\rho \, \vw{\nabla} u } .
\ee
We have chosen the
coefficient relating $\wl$ and $\vw{\nabla} u $ to be strictly positive,
i.e. under the form of an exponential. This is always possible by a suitable
choice of the scalar field $u$.

\subsubsection{Frobenius identity}

Let us take the metric dual of relation (\ref{e:def:wl_rho_u}): it writes
$\uu{\el} = e^\rho \, \wnab u$, or, in terms of components:
\be
    \el_\alpha = e^\rho \, \nabla_\alpha u .
\ee
Taking the covariant derivative, we get
\[
    \nabla_\alpha \el_\beta = e^\rho \nabla_\alpha \rho \nabla_\beta u
                +  e^\rho  \nabla_\alpha \nabla_\beta u
                 = \nabla_\alpha \rho \, \el_\beta + e^\rho  \nabla_\alpha \nabla_\beta u
\]
Antisymmetrizing and using the torsion-free property of $\wnab$ (i.e.
$\nabla_\alpha \nabla_\beta u - \nabla_\beta \nabla_\alpha u = 0$, cf.
Eq.~(\ref{e:bas:torsion-free}) in Appendix~\ref{s:bas}), we get
\be \label{e:def:ext_der_wl_comp}
  \nabla_\alpha \el_\beta - \nabla_\beta \el_\alpha =
  \nabla_\alpha \rho \, \el_\beta -  \nabla_\beta \rho \, \el_\alpha  .
\ee
In the left-hand side there appears the exterior derivative of
the 1-form $\uu{\el}$ (cf. Sec.~\ref{s:bas:ext_deriv} in Appendix~\ref{s:bas}),
while one recognize in the right-hand side the exterior product of
the two 1-forms $\dd\rho$ and $\uu{\el}$. Hence we may rewrite (\ref{e:def:ext_der_wl_comp})
as
\be
    \encadre{ \dd \uu{\el} = \dd\rho \wedge \uu{\el} } .
\ee
This reflects the \defin{Frobenius theorem}\index{Frobenius!theorem}
in its dual formulation (see e.g.
Theorem B.3.2 in Wald's textbook \cite{Wald84}): the exterior derivative of
the 1-form $\uu{\el}$ is the exterior product of $\uu{\el}$ itself with some
1-form ($\dd\rho$ in the present case) if, and only if,
$\uu{\el}$ defines hyperplanes that are integrable in some hypersurface ($\Hor$ in the present case).

\subsubsection{Geodesic generators}

Let us contract the Frobenius identity (\ref{e:def:ext_der_wl_comp}) with $\wl$:
\be \label{e:def:l_contract_Frob}
    \el^\mu \nabla_\mu \el_\alpha - \el^\mu \nabla_\alpha \el_\mu
        = \el^\mu \nabla_\mu \rho \, \el_\alpha
        - \underbrace{\el^\mu \el_\mu}_{0} \nabla_\alpha \rho .
\ee
Now, since $\wl$ is a null vector,
\[
    \el^\mu \nabla_\alpha \el_\mu = \nabla_\alpha (\underbrace{\el^\mu \el_\mu}_{0})
        - \el_\mu \nabla_\alpha \el^\mu ,
\]
from which we get
\[
    \el^\mu \nabla_\alpha \el_\mu = 0 .
\]
Hence (\ref{e:def:l_contract_Frob}) reduces to
\be
    \el^\mu \nabla_\mu \el_\alpha  = \kappa \, \el_\alpha ,
\ee
with
\be
    \kappa := \el^\mu \nabla_\mu \rho  .
\ee


























