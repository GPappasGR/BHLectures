\chapter{The concept of black hole}

\section{Introduction}

\section{General framework}

In these lectures we consider a $n$-dimensional \defin{spacetime}\index{spacetime},
i.e. a pair $(\M, \w{g})$, where $\M$ is a $n$-dimensional smooth manifold
(cf. Sec.~\ref{s:bas:manif} in Appendix~\ref{s:bas}) and $\w{g}$ is a
Lorentzian metric on $\M$ (cf. Sec.~\ref{s:bas:pRiemManif} in
Appendix~\ref{s:bas}). In many parts, the dimension $n$ will be set to 4
--- the standard spacetime value --- but we shall also consider spacetimes with
$n>4$, especially in Chap.~??.

We shall specify further the spacetime structure later on, in particular its
asymptotics and the orientability of some of its parts.


\section{Null hypersurfaces as one-way membranes}

A naive definition of a black hole, involving only words, could be
\begin{quote}
A \emph{black hole}\index{black!hole} is a localized region of spacetime
from which neither massive particles nor light may escape.
\end{quote}
There are essentially two features in this definition: localization
and inescapability. Let us for a moment focus on the latter.
It implies the existence of a \emph{boundary}, which no
(massive or not) particle emitted in the black hole region can cross.
This boundary may be called a
\defin{one-way membrane}\index{one-way membrane}\index{membrane!one-way --},
for it can be crossed from the black hole ``exterior'' towards the black hole
``interior'', but not in the reverse way. The one-way membrane must
a hypersurface of the spacetime manifold $\M$, for it has to divide $\M$ in two regions:
the black hole interior and the black hole exterior.


