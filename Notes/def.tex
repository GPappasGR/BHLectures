\chapter{The concept of black hole}

\minitoc

\section{Introduction}

%%%%%%%%%%%%%%%%%%%%%%%%%%%%%%%%%%%%%%%%%%%%%%%%%%%%%%%%%%%%%%%%%%%%%%%%%%%%%%%%%%%%%%%%

\section{General framework}

\subsection{Spacetime}

In these lectures we consider a $n$-dimensional \defin{spacetime}\index{spacetime},
i.e. a pair $(\M, \w{g})$, where $\M$ is a $n$-dimensional smooth manifold, with $n\geq 2$, and $\w{g}$ is a Lorentzian metric on $\M$. In many parts, the dimension $n$ will be set to 4
--- the standard spacetime value --- but we shall also consider spacetimes with
$n>4$, especially in Chap.~??.

The precise definition and basic properties of a \emph{smooth manifold} are recalled
in Appendix~\ref{s:bas}. Here let us simply say that, in loose terms,
a \defin{manifold}\index{manifold} $\M$ of dimension $n$ is a ``space'' that \emph{locally} resembles $\R^n$,
i.e. can be described by a $n$-tuple of coordinates $(x^1,\ldots,x^n)$. Globally,
$\M$ can be very different from $\R^n$, in particular regarding its topology.

The smooth structure endows the manifold
with the concept of infinitesimal vectors\index{vector!infinitesimal --} $\D\w{x}$, which connect infinitely
close pair of points of $\M$ (cf. Fig.~??). However, for finitely separated points, there is no
longer the concept of connecting vector (contrary for instance to points in $\R^n$).
In other words, vectors on $\M$ do not live in the manifold but in the
tangent spaces $T_p\M$, which are defined at each point $p\in\M$. Each  $T_p\M$
is a $n$-dimensional vector space, which is generated for instance by the infinitesimal displacement
vectors along the $n$ coordinate lines of some coordinate system.

The full definition of the \defin{metric tensor} $\w{g}$ is given in Sec.~\ref{s:bas:pRiemManif} of
Appendix~\ref{s:bas}. The fact that its signature is Lorentzian, i.e.
\be
\mathrm{sign}\; \w{g} = (-,\underbrace{+,\ldots,+}_{\mbox{\small $n-1$ times}}),
\ee
implies that from each point $p\in\M$, there are privileged directions,
which form the so-called \defin{null cones}\index{null!cone} (cf. Fig.~??).
This is an absolute structure of spacetime, independent from any observer.

We shall specify further the spacetime structure later on, in particular its
asymptotics and its orientability.

\subsection{Worldlines}

In relativity, a particle is described by its spacetime extent, which is a smooth curve,
$\Li$ say, and not a point. This curve is called the particle's
\defin{worldline}\index{worldline} and might be thought of as the set of
the ``successive positions'' occupied by the particle as ``time evolves''.
The dynamics of a simple particle (i.e. a particle without internal structure nor
spin) is entirely described by its
\defin{4-momentum}\index{4-momentum} or \defin{energy-momentum vector}\index{energy-momentum!vector}\footnote{When $n\not=4$, \emph{energy-momentum vector} is definitely a better name
than \emph{4-momentum}!}, which is a vector field $\w{p}$ defined along $\Li$,
tangent to $\Li$ at each point and future-directed (cf. Fig.~??).

One distinguishes two types of particles:
\begin{itemize}
\item the \defin{massive particles}\index{massive!particle}\index{particle!massive --}, for which
$\Li$ is a timelike curve, or equivalently, for which
$\w{p}$ is a timelike vector:
\be
    \w{g}(\w{p},\w{p}) = \w{p}\cdot\w{p} < 0 ;
\ee
one says then that $\Li$ is a timelike curve
\item the \defin{massless particles}\index{massless!particle}\index{particle!massless --},
such as the photon,
for which $\Li$ is a null curve, or equivalently, for which  $\w{p}$ is a null vector:
\be
    \w{g}(\w{p},\w{p}) = \w{p}\cdot\w{p} = 0 .
\ee
\end{itemize}
In both cases, the \defin{mass} of the particle is defined by
\be \label{e:def:def_mass}
   m = \sqrt{- \w{p}\cdot\w{p}} .
\ee
Of course, for a massless particle, we get $m=0$.

If the particle feels only gravitation, i.e. if no non-gravitational force
is exerted on it, the energy-momentum vector must be a
\defin{geodesic vector}\index{geodesic!vector}, i.e. it obeys
\be \label{e:def:p_geodesic}
    \encadre{\wnab_{\w{p}}\,  \w{p} = 0 } ,
\ee
or, in terms of components
\be
    p^\mu \nabla_\mu p^\alpha = 0 .
\ee
This implies that the worldline $\Li$ must be a
\defin{geodesic}\index{geodesic} of the spacetime $(\M,\w{g})$.
\begin{remark} \label{r:def:geodesic_vector}
The reverse is not true, i.e. having $\Li$ geodesic and $\w{p}$
tangent to $\Li$ does not imply (\ref{e:def:p_geodesic}), but the
weaker condition $\wnab_{\w{p}}\,  \w{p} = \alpha \, \w{p}$, with $\alpha$
a scalar field along $\Li$.
\end{remark}
For massive particles, Eq.~(\ref{e:def:p_geodesic}) can be derived from
a variational principle, the action being simply the worldline's length
as given by the metric tensor:
\be
    S = \int_A^B \D s = \int_{\lambda_A}^{\lambda_B}
     \sqrt{-\w{g}\left(\frac{\D\w{x}}{\D\lambda}, \frac{\D\w{x}}{\D\lambda} \right)}
     \, \D\lambda .
\ee
For photons, Eq.~(\ref{e:def:p_geodesic}) can be derived from the Maxwell equations
within the geometrical optics approximation, with the assumption that
the photon energy-momentum vector is related to the wave 4-vector $\w{k}$ by
\be \label{e:def:p_hbar_k}
    \w{p} = \hbar \w{k} .
\ee

\subsubsection{Massive particles}

For a massive particle, the constraint of having $\Li$ timelike
has a simple geometrical meaning: the worldline $\Li$ must
always lie inside the light cones of events along $\Li$ (cf. Fig.~??).
The fondamental link between physics and geometry is that the
\defin{proper time}\index{proper!time}\index{time!proper --} $\tau$ of the particle
is the metric length along the worldline, increasing towards the future:
\be \label{e:def:proper_time}
    \D\tau = \sqrt{- \w{g}(\D\w{x}, \D\w{x})} = \sqrt{- g_{\mu\nu} \D x^\mu \, \D x^\nu} ,
\ee
where $\D\w{x}$ is an infinitesimal future-directed displacement
along $\Li$.

The particle's \defin{4-velocity}\index{4-velocity} is defined the vector field
$\w{u}$
tangent to the worldline $\Li$ associated with the parametrization of
$\Li$ by the proper time:
\be \label{e:def:def_u}
    \encadre{ \w{u} := \frac{\D\w{x}}{\D\tau} }.
\ee
By construction, this is a unit timelike vector:
\be \label{e:def:u_unit}
    \w{u}\cdot\w{u} = -1 .
\ee
For a simple particle (no internal structure),
the 4-momentum $\w{p}$ is tangent to $\Li$; it is then necessarily
colinear to $\w{u}$. Since both vectors are future-directed,
Eqs.~(\ref{e:def:def_mass}) and (\ref{e:def:u_unit}) lead to
\be \label{e:def:p_m_u}
    \w{p} = m\, \w{u} .
\ee

\subsubsection{Massless particles (photons)}

For a massless particle, Eq.~(\ref{e:def:proper_time}) would lead to $\D\tau=0$
since $\D\w{x}$ would be a null vector. There is then no natural parameter
along a null geodesic. However, one can single out a whole family of them,
called \emph{affine parameters}. An \defin{affine parameter}\index{affine!parameter}
along a null geodesic $\Li$ is a parameter $\lambda$ such that
the associated tangent vector,
\be
    \w{v} := \frac{\D\w{x}}{\D\lambda} ,
\ee
is a geodesic vector field: $\wnab_{\w{v}} \, \w{v} = 0$. In general,
the tangent vector associated to a given parameter fullfils only
$\wnab_{\w{v}} \, \w{v} = \alpha \, \w{v}$, with $\alpha$ a scalar field
along $\Li$ (cf. Remark~\ref{r:def:geodesic_vector} above).

The qualifier \emph{affine} arises from the fact any two affine parameters
$\lambda$ and $\lambda'$ are related by an affine transformation:
\be \label{e:def:affine_transf}
    \lambda' = a \lambda + b,
\ee
with $a$ and $b$ two constants.
Given that the photon energy-momentum vector $\w{p}$ is a geodesic vector
[Eq.~(\ref{e:def:p_geodesic})],
a natural choice of the affine parameter $\lambda$ is that associated with
$\w{p}$:
\be \label{e:def:p_dxdl}
    \w{p} = \frac{\D\w{x}}{\D\lambda} .
\ee
This fixes $a=1$ in the transformation (\ref{e:def:affine_transf}).

\subsection{Quantities measured by an observer}

In the simplest modelization, an \defin{observer}\index{observer} $\Obs$
is described by a timelike worldline $\Li_{\Obs}$ in the spacetime $(\M,\w{g})$.
Let us suppose that the observer encounters a particle at some event $A$.
Geometrically, this means that the worldline $\Li$ of the particle
intersects $\Li_{\Obs}$ at $A$. Then, the \defin{energy}\index{energy!of a particle}
$E$ and the \defin{momentum}\index{momentum!of a particle} $\w{P}$ of the
particle, both measured by $\Obs$, are given by the orthogonal decomposition of the
particle's energy-momentum vector $\w{p}$ with respect to $\Li_{\Obs}$:
\be \label{e:def:p_E_P}
    \encadre{ \w{p} = E \w{u}_{\Obs} + \w{P} },\quad\mbox{with}\quad
        \w{u}_{\Obs}\cdot\w{P} = 0 ,
\ee
where $\w{u}_{\Obs}$ is the 4-velocity of observer $\Obs$, i.e. the future-directed
unit tangent vector to $\Li_{\Obs}$.
By taking the scalar product of Eq.~(\ref{e:def:p_E_P}) with $\w{u}_{\Obs}$,
we obtain the following expressions for $E$ and $\w{P}$:
\be \label{e:def:E_obs}
    \encadre{E = - \w{u}_{\Obs}\cdot\w{p}}
\ee
\be
    \encadre{\w{P} = \w{p} + (\w{u}_{\Obs}\cdot\w{p})\, \w{u}_{\Obs}} .
\ee
The scalar square of Eq.~(\ref{e:def:p_E_P}) leads to
\be
    \underbrace{\w{p}\cdot\w{p}}_{-m^2} = E^2
    \underbrace{\w{u}_{\Obs}\cdot\w{u}_{\Obs}}_{-1} + 2 E \underbrace{\w{u}_{\Obs}\cdot\w{P}}_{0}
    + \w{P}\cdot\w{P} ,
\ee
where we have used Eq.~(\ref{e:def:def_mass}) to let appear the particle's mass
$m$. Hence we recover Einstein's relation\index{Einstein!relation}:
\be \label{e:def:E2_m2_P2}
    \encadre{E^2 = m^2 + \w{P}\cdot\w{P} }.
\ee

An infinitesimal displacement $\D\w{x}$ of the particle along its worldline
is related to the energy-momentum vector $\w{p}$ by
\be \label{e:def:dx_p_dl}
    \D\w{x} = \w{p} \, \D\lambda,
\ee
where $\lambda$ is the affine parameter along the particle's worldline
whose tangent vector is $\w{p}$ [cf. Eq.~(\ref{e:def:p_dxdl}) for a massless
particle and Eqs.~(\ref{e:def:def_u}) and (\ref{e:def:p_m_u}) with
$\lambda := \tau/m$ for a massive particle]. Substituting (\ref{e:def:p_E_P})
for $\w{p}$ in (\ref{e:def:dx_p_dl}), we get the orthogonal decomposition
of $\D\w{x}$ with respect to $\Li_{\Obs}$:
\be
    \D\w{x} = E \D\lambda \, \w{u}_{\Obs} + \D\lambda\,  \w{P} .
\ee
$\Obs$'s proper time elapsed during the particule displacement is the
coefficient in front of $\w{u}_{\Obs}$: $\D\tau_{\Obs} = E \D\lambda$ and the
the particle's displacement in $\Obs$'s rest frame is the part orthogonal
to $\w{u}_{\Obs}$: $\D\w{X} = \D\lambda\,  \w{P}$ (cf. Fig.~??). By definition,
the particle's velocity with respect to $\Obs$ is
\be
    \w{V} := \frac{\D\w{X}}{\D\tau_{\Obs}} = \frac{\D\lambda\,  \w{P}}{E \D\lambda}.
\ee
Hence the relation
\be \label{e:def:P_E_V}
    \encadre{ \w{P} = E \, \w{V}} .
\ee

The above relations are valid for any kind of particle, massive or not.
For a massive particle, the energy-momentum vector $\w{p}$ is related to the
particle's 4-velocity $\w{u}$ via (\ref{e:def:p_m_u}). Inserting this relation
into (\ref{e:def:E_obs}), we obtain
\be \label{e:def:E_Gam_m}
    E = \Gamma \, m,
\ee
where
\be
    \Gamma := - \w{u}_{\Obs}\cdot\w{u}
\ee
is the \defin{Lorentz factor}\index{Lorentz!factor} of the particle with respect
to the observer. If we depart from units with $c=1$, Eq.~(\ref{e:def:E_Gam_m})
becomes the famous relation $E = \Gamma m c^2$.
Combining (\ref{e:def:P_E_V}) and (\ref{e:def:E_Gam_m}) yields also a familiar
relation:
\be \label{e:def:P_Gam_m_V}
    \w{P} = \Gamma m \, \w{V} .
\ee
Finaly, inserting (\ref{e:def:E_Gam_m}) and (\ref{e:def:P_Gam_m_V}) into
(\ref{e:def:E2_m2_P2}) leads to the well-known expression of the Lorentz factor in
terms of the velocity:
\be
    \Gamma = \left( 1 - \w{V}\cdot\w{V} \right) ^{-1/2} .
\ee

For a massless particle (photon), inserting (\ref{e:def:P_E_V}) into the
Einstein relation (\ref{e:def:E2_m2_P2}) with $m=0$ yields
\be
    \w{V}\cdot\w{V} = 1 .
\ee
This means that the norm of the velocity of the massless particle with respect to $\Obs$
is the speed of light $c$ ($=1$ in our units).
For a photon associated with a monochromatic radiation, the wave 4-vector $\w{k}$
admits the following orthogonal decomposition:
\be
    \w{k} = \omega \left(\w{u} + \w{V} \right) ,
\ee
where $\omega = 2\pi \nu$ and $\nu$ is the radiation frequency as measured by
observer $\Obs$. In view of (\ref{e:def:p_hbar_k}) and (\ref{e:def:p_E_P}), we
get the Planck-Einstein relation\index{Planck-Einstein relation}:
\be
    E = h \nu .
\ee

\subsection{Einstein equation}

Saying that gravitation in the spacetime $(\M, \w{g})$ is ruled by
\defin{general relativity}\index{general!relativity} amounts
to demanding that the metric $\w{g}$ obeys \defin{Einstein equation}\index{Einstein!equation}:
\be \label{e:bas:Einstein_eq}
    \encadre{ \w{R} - \frac{1}{2}\, R\, \w{g} + \Lambda\, \w{g} = 8\pi \w{T} },
\ee
where $\w{R}$ is the Ricci tensor of $\w{g}$, $R$ the Ricci scalar of $\w{g}$
(cf. Sec.~\ref{s:bas:Ricci_tensor} in Appendix~\ref{s:bas}), $\Lambda$ some
constant, called the \defin{cosmological constant}\index{cosmological!constant}
and $\w{T}$ is the energy-momentum tensor\index{energy-momentum tensor} of
matter and non-gravitational fields.

By taking the trace of (\ref{e:bas:Einstein_eq}) with respect to $\w{g}$, it is
easy to show that the Einstein equation (\ref{e:bas:Einstein_eq}) is
equivalent to
\be \label{e:bas:Einstein_eq_n}
    \encadre{ \w{R}  = \frac{2}{n-2}\,\Lambda\,  \w{g}
    + 8\pi \left( \w{T} - \frac{1}{n-2}\,  T \, \w{g} \right) },
\ee
where $T := g^{\mu\nu} T_{\mu\nu}$ is the trace of $\w{T}$ with respect to
$\w{g}$.

\begin{remark}
The dimension $n$ of the spacetime does not appear in the Einstein equation
(\ref{e:bas:Einstein_eq}); on the contrary, the variant
(\ref{e:bas:Einstein_eq_n}) depends on $n$.
\end{remark}

%%%%%%%%%%%%%%%%%%%%%%%%%%%%%%%%%%%%%%%%%%%%%%%%%%%%%%%%%%%%%%%%%%%%%%%%%%%%%%%%%%%%%%%%

\section{A first definition of black holes}

\subsection{Black holes and null hypersurfaces} \label{s:def:first_defin}

A naive definition of a black hole, involving only words, could be
\begin{quote}
A \defin{black hole}\index{black!hole} is a localized region of spacetime
from which neither massive particles nor massless ones (photons) may escape.
\end{quote}
There are essentially two features in this definition: \emph{localization}
and \emph{inescapability}. Let us for a moment focus on the latter.
It implies the existence of a \emph{boundary}, which no
particle emitted in the black hole region can cross.
This boundary is called the
\defin{event horizon}\index{event!horizon}\index{horizon!event --} and is
quite often referred to simply as the \defin{horizon}.
It is a \defin{one-way membrane}\index{one-way membrane}\index{membrane!one-way --},
in the sense that it can be crossed from the black hole ``exterior'' towards
the black hole region, but not in the reverse way. The one-way membrane must be
a hypersurface of the spacetime manifold $\M$, for it has to divide $\M$ in two regions:
the interior (the black hole itself) and the exterior region.
Let us recall that a hypersurface is a submanifold of $\M$ of codimension 1
(cf. Sec.~\ref{s:bas:embed} in Appendix~\ref{s:bas}).

To discuss further which hypersurface could act as a black hole boundary,
one should recall that, on a Lorentzian manifold $(\M,\w{g})$, there are
three classes of hypersurfaces. The classification
depends on the type of metric induced by $\w{g}$ on the
hypersurface, $\Sigma$ say, the
\defin{induced metric}\index{induced!metric}\index{metric!induced --} being
nothing but the restriction $\left.\w{g}\right| _{\Sigma}$ of $\w{g}$
to vector fields tangent to $\Sigma$.
A hypersurface $\Sigma$ is said to be
\begin{itemize}
\item \defin{spacelike} iff $\left.\w{g}\right| _{\Sigma}$ is positive definite,
i.e. iff $\mathrm{sign} \left.\w{g}\right| _{\Sigma} = (+,+,+)$,
i.e. iff $(\Sigma,  \left.\w{g}\right| _{\Sigma})$ is a Riemannian manifold;
\item \defin{timelike} iff $\left.\w{g}\right| _{\Sigma}$ is a Lorentzian metric,
i.e. iff $\mathrm{sign} \left.\w{g}\right| _{\Sigma} = (-,+,+)$,
i.e. iff $(\Sigma,  \left.\w{g}\right| _{\Sigma})$ is a Lorentzian manifold;
\item \defin{null} iff $\left.\w{g}\right| _{\Sigma}$ is degenerate\footnote{
Cf. Sec.~\ref{s:bas:metric} in Appendix~\ref{s:bas} for the definition of a
degenerate bilinear form; the degeneracy
implies that the bilinear form $\left.\w{g}\right| _{\Sigma}$ is not,
strictly speaking, a metric on $\Sigma$.}
i.e. iff $\mathrm{sign} \left.\w{g}\right| _{\Sigma} = (0,+,+)$.
\end{itemize}
The hypersurface type can also be deduced from the normal vector
$\w{n}$ to it (cf. Sec.~\ref{s:bas:hyp_normal}):
\begin{itemize}
\item $\Sigma$ spacelike $\iff$ $\w{n}$ timelike;
\item $\Sigma$ timelike $\iff$ $\w{n}$ spacelike;
\item $\Sigma$ null $\iff$ $\w{n}$ null.
\end{itemize}
These equivalences are easily proved by considering a $\w{g}$-orthogonal basis
adapted to $\Sigma$.

\begin{remark}
Null hypersurfaces have the distinctive feature that their normals are
also tangent to them. Indeed, by definition, the normal $\w{n}$ is null iff
$\w{n}\cdot\w{n}=0$, which is nothing but the condition
for $\w{n}$ to be tangent to $\Sigma$.
\end{remark}

Only null and spacelike hypersurfaces are eligible as a one-way membrane
regarding timelike and null worldlines (cf. Fig.~??).
The limit case between two-way membranes (timelike hypersurfaces)
and one-way ones being null hypersurfaces, it is quite natural to select the
latter ones for the black hole boundary, rather than spacelike hypersurfaces.
Note however that in Chap.~??, we shall see that spacelike hypersurfaces,
called \emph{dynamical
horizons}\index{dynamical!horizon}\index{horizon!dynamical --}, are involved
in the quasi-local approaches to black holes.

\subsection{Geometry of null hypersurfaces}

Having decided that the black hole event horizon must be a null hypersurface,
let us examine the geometrical properties of such a hypersurface. We shall
denote it by $\Hor$, for \emph{horizon}, but the results of this section
will be valid for any null hypersurface.


As any hypersurface, $\Hor$ can be locally considered as a level set,
around any point of $\Hor$, there exists an open subset $\mathscr{U}$
of $\M$ (possibly  $\mathscr{U} = \M$) and
a smooth scalar field $u:\ \mathscr{U} \rightarrow \R$ such that
\be \label{e:def:Hor_u_zero}
    \forall p \in \mathscr{U},\quad p\in \Hor \iff u(p) = 0 .
\ee
For example, if $\Hor$ is a light cone in Minkowsky spacetime and
$(t,r,\th,\ph)$ are standard spherical coordinates, we may choose
the retarded time coordinate $u=t-r$.

Let $\wl$ be a vector field normal to $\Hor$. Since $\Hor$ is a null hypersurface,
$\wl$ is a null vector:
\be \label{e:def:wl_null}
    \wl\cdot\wl = 0
\ee
\begin{remark}
As a consequence of (\ref{e:def:wl_null}), there is no natural normalization
of $\wl$, contrary to the case of timelike or spacelike hypersurfaces,
where one can always choose the normal to be a unit vector
(scalar square equal to $1$ or $-1$). It follows that there is no unique choice
of $\wl$. At this stage, any rescaling $\wl \mapsto \wl' =  \alpha \wl$, with
$\alpha$ a strictly positive (or strictly negative) scalar field on $\Hor$,
yields a normal vector field $\wl'$ as valid as $\wl$.
\end{remark}
The null normal vector field $\wl$ is a priori defined on $\Hor$
only and not at points $p\not\in\Hor$.
However, it is worth to consider $\wl$ as a vector field
not confined to $\Hor$ but defined
in some open subset of $\M$ around $\Hor$.
In particular this would permit to define the spacetime covariant
derivative $\w{\nabla}\wl$, which is not possible if the
support of $\wl$ is restricted to $\Hor$.
Following Carter \cite{Carte97}, a simple way to achieve
this is to consider not only a single null hypersurface $\Hor$,
but a foliation of $\M$ (in the vicinity
of $\Hor$) by a family of null hypersurfaces, such that $\Hor$ is an
element of this family.
Without any loss of generality,
we may select the value of the scalar field $u$ to label these hypersurfaces and
denote the family by $(\Hor_u)$. The null hypersurface $\Hor$
is then nothing but the element $\Hor = \Hor_{u=0}$ of this family
[Eq.~(\ref{e:def:Hor_u_zero})].
The vector field $\wl$ can then be viewed as defined in the part of $\M$
foliated by $(\Hor_u)$, such that at each point in this region, $\wl$
is null and normal to $\Hor_u$ for some value of $u$.

Obviously the family $(\Hor_u)$ is non-unique but all geometrical
quantities that we shall introduce hereafter do not depend upon the choice
of the foliation $\Hor_u$ once they are evaluated at $\Hor$.

Since $\Hor$ is a hypersurface where $u$ is constant [Eq.~(\ref{e:def:Hor_u_zero})],
we have, by definition,
\bea
    \forall \w{v}\in T_p\M,\quad \w{v} \mbox{\ tangent to\ }\Hor & \iff  & \langle \wnab u , \w{v} \rangle = 0 \nonumber \\
    & \iff & \vw{\nabla} u \cdot \w{v} = 0 ,   \label{e:def:nab_u_normal}
\eea
where $\vw{\nabla} u$ is the gradient vector field of the scalar field $u$.
Property (\ref{e:def:nab_u_normal}) means that $\vw{\nabla} u$ is
a normal vector field to $\Hor$. By uniqueness of the normal direction, it
must then be colinear to $\wl$. Therefore, there must exist some scalar
field $\rho$ such that
\be \label{e:def:wl_rho_u}
    \encadre{\wl = e^\rho \, \vw{\nabla} u } .
\ee
We have chosen the
coefficient linking $\wl$ and $\vw{\nabla} u $ to be strictly positive,
i.e. under the form of an exponential. This is always possible by a suitable
choice of the scalar field $u$.

\subsubsection{Frobenius identity}

Let us take the metric dual of relation (\ref{e:def:wl_rho_u}): it writes
$\uu{\el} = e^\rho \, \wnab u$, or, in terms of components:
\be
    \el_\alpha = e^\rho \, \nabla_\alpha u .
\ee
Taking the covariant derivative, we get
\[
    \nabla_\alpha \el_\beta = e^\rho \nabla_\alpha \rho \nabla_\beta u
                +  e^\rho  \nabla_\alpha \nabla_\beta u
                 = \nabla_\alpha \rho \, \el_\beta + e^\rho  \nabla_\alpha \nabla_\beta u
\]
Antisymmetrizing and using the torsion-free property of $\wnab$ (i.e.
$\nabla_\alpha \nabla_\beta u - \nabla_\beta \nabla_\alpha u = 0$, cf.
Eq.~(\ref{e:bas:torsion-free}) in Appendix~\ref{s:bas}), we get
\be \label{e:def:ext_der_wl_comp}
  \nabla_\alpha \el_\beta - \nabla_\beta \el_\alpha =
  \nabla_\alpha \rho \, \el_\beta -  \nabla_\beta \rho \, \el_\alpha  .
\ee
In the left-hand side there appears the exterior derivative of
the 1-form $\uu{\el}$ (cf. Sec.~\ref{s:bas:ext_deriv} in Appendix~\ref{s:bas}),
while one recognize in the right-hand side the exterior product of
the two 1-forms $\dd\rho$ and $\uu{\el}$. Hence we may rewrite (\ref{e:def:ext_der_wl_comp})
as
\be
    \encadre{ \dd \uu{\el} = \dd\rho \wedge \uu{\el} } .
\ee
This reflects the \defin{Frobenius theorem}\index{Frobenius!theorem}
in its dual formulation (see e.g.
Theorem B.3.2 in Wald's textbook \cite{Wald84}): the exterior derivative of
the 1-form $\uu{\el}$ is the exterior product of $\uu{\el}$ itself with some
1-form ($\dd\rho$ in the present case) if, and only if,
$\uu{\el}$ defines hyperplanes that are integrable in some hypersurface ($\Hor$ in the present case).

\subsubsection{Geodesic generators}

Let us contract the Frobenius identity (\ref{e:def:ext_der_wl_comp}) with $\wl$:
\be \label{e:def:l_contract_Frob}
    \el^\mu \nabla_\mu \el_\alpha - \el^\mu \nabla_\alpha \el_\mu
        = \el^\mu \nabla_\mu \rho \, \el_\alpha
        - \underbrace{\el^\mu \el_\mu}_{0} \nabla_\alpha \rho .
\ee
Now, since $\wl$ is a null vector,
\[
    \el^\mu \nabla_\alpha \el_\mu = \nabla_\alpha (\underbrace{\el^\mu \el_\mu}_{0})
        - \el_\mu \nabla_\alpha \el^\mu ,
\]
from which we get
\[
    \el^\mu \nabla_\alpha \el_\mu = 0 .
\]
Hence (\ref{e:def:l_contract_Frob}) reduces to
\be \label{e:def:wl_geod_kappa_dual}
    \el^\mu \nabla_\mu \el_\alpha  = \kappa \, \el_\alpha ,
\ee
with
\be
    \kappa := \el^\mu \nabla_\mu \rho = \wnab_{\wl}\,  \rho .
\ee
The metric dual of (\ref{e:def:wl_geod_kappa_dual}) is
\be \label{e:def:wl_geod_kappa}
    \encadre{ \wnab_{\wl}\, \wl = \kappa \, \wl } .
\ee
This equation implies that the field lines of $\wl$ are geodesics.
To demonstrate this, we note that a rescaling
\be \label{e:def:wl_rescale}
    \wl \mapsto \wl' =  \alpha \wl
\ee
with $\alpha$ a strictly positive scalar field can be performed to yield
a geodesic vector field\index{geodesic!vector field} $\wl'$, i.e.
a vector field that obeys
\be
    \wnab_{\wl'}\, \wl' = 0 .
\ee
Indeed, Eqs.~(\ref{e:def:wl_rescale}) and
(\ref{e:def:wl_geod_kappa}) imply
\[
    \wnab_{\wl'}\, \wl' = \alpha\left(
        \wnab_{\wl}\, \alpha + \kappa \alpha \right)
\]
Hence, since $\alpha>0$,
\[
    \wnab_{\wl'}\, \wl' = 0  \iff  \wnab_{\wl}\, \ln \alpha = -\kappa .
\]
Therefore it suffices to solve $\wnab_{\wl}\, \ln \alpha = -\kappa$, which
is a first-order ordinary differential equation along each field line of $\wl$
to ensure that $\wl'$ is a geodesic vector field.
The field lines of $\wl'$ are then null geodesics and $\wl'$ is the tangent
vector to them associated with some affine parameter $\lambda$.
On the other side, if $\kappa\not=0$, $\wl$ is not a geodesic vector field
and therefore cannot be associated with some affine parameter. For this
reason the quantity $\kappa$ is called the
\defin{non-affinity parameter}\index{non-affinity parameter} of
the null normal $\wl$.

Since $\wl$ is colinear to $\wl'$, it obviously share the same field lines.
These field lines are called the
\defin{null generators}\index{null!generator}\index{generator!of a null hypersurface}
of the hypersurface $\Hor$.

Hence, we have shown that
\begin{quote}
Any null hypersurface $\Hor$ is ruled by a family of null geodesics, called the
null generators, and each vector field $\wl$ normal to $\Hor$ is
tangent to these null geodesics.
\end{quote}

\subsection{Expansion of null hypersurfaces}

Let us now focus on the first aspect of the black hole definition given
in Sec.~\ref{s:def:first_defin}: \emph{localization}.
This feature is crucial to distinguish a black hole boundary from other types
of null hypersurfaces. For instance the interior of a future null cone
in Minkowski spacetime is a region from which no particle may escape,
but since the null cone is expanding, particles can travel arbitrary far from
the center. Therefore a null cone does not define a black hole.
Let us then consider the expansion of null hypersurfaces.

To encompass the idea that an event horizon delimitates a
region of spacetime of compact spacelike sections, we shall assume
that the boundary of these sections have the topology of a sphere, more
precisely the $(n-2)$-dimensional sphere $\mathbb{S}^{n-2}$ in a $n$-dimensional
spacetime. The topology of $\Hor$ is then that of a ``tube'' or ``cylinder'':
\be
    \Hor \simeq \R \times \mathbb{S}^{n-2}.
\ee
For the standard spacetime dimension ($n=4$), this is of course
$\Hor \simeq \R \times \mathbb{S}^{2}$.

To give some concretness at the ``spacelike sections'' of $\Hor$, let us
assume that





















