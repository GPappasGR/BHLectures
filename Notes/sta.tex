\chapter{Stationary black holes}
\label{s:sta}

\minitoc

\section{Introduction}

\section{Definition and first properties} \label{s:sta:def_station}

Let us consider a spacetime $(\M,\w{g})$ that contains a black hole, as defined in
Sec.~\ref{s:glo:def_BH}. In particular, $(\M,\w{g})$ admits a future null
infinity $\scri^+$ and a past null infinity $\scri^-$. One says that
$(\M,\w{g})$ is a \defin{stationary spacetime}\index{stationary!spacetime}
if (i) it is invariant under
the action of the translation group $(\mathbb{R},+)$ and (ii) the orbits of
the group action are timelike curves in the vicinity of $\scri^+$ and
$\scri^-$. It is equivalent to say that there exists a Killing vector field
$\w{\xi}$ (the generator of the translation group) that is timelike in the vicinity of $\scri^+$ and $\scri^-$.

\begin{remark}
Some authors (e.g. Carter \cite{Carte73b}) call such spacetimes
\emph{pseudo-stationary}\index{pseudo-stationary}, keeping the qualifier
\emph{stationary} for the case where the Killing field $\w{\xi}$ is timelike
in all $\M$. As we going to see, when $\M$
contains a black hole, $\w{\xi}$ cannot be timelike everywhere,
so only \emph{pseudo-stationarity} in the above sense is relevant for them.
Our terminology follows that of Chru\'sciel, Lopes Costa \& Heusler \cite{ChrusLH12}
and Choquet-Bruhat \cite{Choqu09}.
\end{remark}

If $(\M,\w{g})$ is invariant under the action of the isometry group $(\mathbb{R},+)$,
so is $\scri^+$ (under some proper extension of $\w{\xi}$ to $\tilde{\M}$)
and therefore its causal past $J^-(\scri^+)$. As the boundary of $J^-(\scri^+)$
inside $\M$, the event horizon $\Hor$ must therefore be invariant under the
action of the isometry group.
Note that this means that $\Hor$ is invariant \emph{as a whole}, not that
each point of $\Hor$ is invariant under the group action.
Now, $\Hor$ is globally invariant if, and only if, the
generator $\w{\xi}$ of the isometry group is tangent to $\Hor$.
Let us assume that $\Hor$ is smooth (which sounds likely in a stationary context;
a rigorous proof can be found in \cite{ChrusDGH01}),
it is then a null hypersurface (Property~4 in Sec.~\ref{s:glo:properties_H}).
Since a timelike vector cannot be tangent to a null hypersurface (cf. the
lemma in Sec.~\ref{s:def:spacelike_sections}), we conclude that
\begin{greybox}
The stationary Killing vector field  $\w{\xi}$ is tangent to the event horizon
$\Hor$, which implies that $\w{\xi}$ is either null or spacelike on $\Hor$.
\end{greybox}

\section{The event horizon as a Killing horizon}

Let us discuss successively the two allowed types for the stationary
Killing vector $\w{\xi}$ on $\Hor$: null and spacelike.

\subsection{Null stationary Killing field on $\Hor$: the staticity theorem}

By the lemma of Sec.~\ref{s:def:spacelike_sections}, if the Killing vector
field $\w{\xi}$ is null on $\Hor$, it is necessarily tangent to the null geodesic generators
of $\Hor$ and therefore collinear to the null normals $\wl$ of $\Hor$. From the definition
given in Sec.~\ref{s:def:def_Killing_hor}, it follows immediately that
$\Hor$ is a Killing horizon (with respect to the Killing field $\w{\xi}$).
In dimension $n=4$ and using the Einstein equation,
D.~Sudarski and R.M.~Wald (1992) \cite{SudarW92} have then proven that $\w{\xi}$ must be
hypersurface-orthogonal everywhere, i.e. that the spacetime $(\M,\w{g})$ is \defin{static}\index{static spacetime}. For this reason, Sudarski \& Wald's result is often
called the \defin{staticity theorem}\index{staticity theorem}.

Having that $(\M,\w{g})$ is static, we can go further and
apply the
\begin{greybox}[frametitle={Israel uniqueness theorem:}]
\index{Israel uniqueness theorem}
If $(\M,\w{g})$ is a $n$-dimensional static spacetime
containing a black hole, with $\w{g}$ solution of the vacuum Einstein
equation, then the domain of outer communications of $\M$ is isomorphic
to the domain of outer communications of a $n$-dimensional Schwarzschild spacetime\index{Schwarzschild!spacetime}.
\end{greybox}
This theorem has been proved in 1967 by W.~Israel \cite{Israe67},
and improved latter by many authors (in particular by
P. Chru\'sciel \& G. Galloway (2010) \cite{ChrusG10}, who removed
the hypothesis of analyticity).
A demonstration of Israel's theorem can be found in Straumann's textbook \cite{Strau04}.

So basically, in dimension $n=4$ (i.e. when the staticity theorem applies), all stationary vacuum black holes with the stationary Killing field $\w{\xi}$ null
on $\Hor$ are nothing but Schwarzschild black holes, which we will study in detail in Chap.~\ref{s:sch}.


\subsection{Spacelike stationary Killing field on $\Hor$: the strong rigidity theorem}
\label{s:sta:strong_rigidity}

When $\w{\xi}$ is spacelike on $\Hor$, it obviously cannot be collinear to
any null normal $\wl$.
Assuming that $\Hor$ has cross-sections of spherical topology, we observe
that, with respect to the null geodesic generators of $\Hor$, the field lines of $\w{\xi}$
form some helices (cf. Fig.~??). By reciprocity, with respect to the field lines of $\w{\xi}$,
the null geodesic generators form some helices as well (cf. Fig.~??).
Since asymptotically the field lines of $\w{\xi}$ are worldlines of inertial observers,
we may say (in loose terms at this stage) that the event horizon $\Hor$
``is rotating'', all the more that we have seen above that when the null
generators coincide with the field lines of $\w{\xi}$, the black hole is static, i.e. non-rotating.

Since the Killing field $\w{\xi}$ is not null on $\Hor$, we cannot say a priori
that $\Hor$ is a Killing horizon. However, it turns out
that this is indeed the case, according to a famous result by S.W.~Hawking (1972)
\cite{Hawki72,HawkiE73}, known as the
\defin{strong rigidity theorem}\index{strong!rigidity theorem}\index{rigidity theorem!strong --}.
Assuming $n=4$ and the metric $\w{g}$ obeying the vacuum Einstein equation,
Hawking was able to show that there exists a second Killing vector field,
$\w{\chi}$ say, which is null on $\Hor$. Hence $\Hor$ is a Killing horizon
in this case as well, albeit not with respect to the
stationary Killing vector field $\w{\xi}$.

Hawking's result has been extended to dimensions $n\geq 4$ by V.~Moncrief
and J.~Isenberg (2008) \cite{MoncrI08}, under the hypotheses that $\Hor$
has cross-sections that are compact and transverse to $\w{\xi}$ (see also
Theorem~8.1 p.~470 of Choquet-Bruhat's textbook \cite{Choqu09}).
Both Hawking's result
and Moncrief \& Isenberg's one rely on the rather strong assumption that $\M$ and $\Hor$
are (real) \emph{analytic} manifolds, with $\w{g}$ being an analytic field. On physical grounds,
it would be desirable to assume only \emph{smooth} manifolds and fields.
Recently, S.~Alexakis, A.D.~Ionescu and S.~Klainerman \cite{AlexaIK14} (2014)
have succeeded in proving the strong rigidity theorem without the analyticity
assumption, but only for slowly rotating black holes.

Since we have two Killing vectors, $\w{\xi}$ and $\w{\chi}$, we may
form any linear combination of them with constant coefficients
and still get a Killing vector. For instance, if $\Omega_H$ is a non-zero constant,
the vector field $\w{\eta}$ defined by
\be
    \w{\eta} = \frac{1}{\Omega_H} \left( \w{\chi} - \w{\xi} \right)
    \quad\iff\quad
    \w{\chi} = \w{\xi} + \Omega_H \w{\eta} ,
\ee
is a Killing vector field on $\M$.
One can show (see e.g. \cite{Chrus97} for a rigorous proof) that $\Omega_H$
and some constant rescaling of $\w{\chi}$
can be chosen so that $\w{\eta}$ is a spacelike vector field whose
field lines are closed, with $2\pi$-periodicity in terms of the parameter $\ph$
associated to $\w{\eta}$ (i.e. $\w{\eta} = \D/\D\ph$ along the field lines),
and such that $\w{\eta}$ vanishes on a timelike 2-dimensional surface, called
the \defin{rotation axis}\index{rotation!axis}.
It follows that
the isometry group whose generator is $\w{\eta}$ is the rotation group
$\mathrm{SO}(2)$. In other words, the spacetime $(\M,\w{g})$ is
\defin{axisymmetric}\index{axisymmetric!spacetime} in addition to be stationary.
The constant $\Omega_H$ is then called the
\defin{black hole rotation velocity}\index{black hole!rotation velocity}\index{rotation!velocity}.

By the very definition of stationarity, the Killing vector field $\w{\xi}$ is
timelike in the vicinity of $\scri^+$ and $\scri^-$. If $\w{\xi}$ is spacelike
on $\Hor$, as assumed in this section, by continuity it must be spacelike
in some part of the domain of outer communications $\langle\langle \M\rangle\rangle$
near $\Hor$. The simplest configuration is then when
$\w{\xi}$ is spacelike in some connected region $\mathscr{G}\subset \langle\langle \M\rangle\rangle$
around $\Hor$, null at the boundary of $\mathscr{G}$ and timelike outside $\mathscr{G}$
up to $\scri^+$ and $\scri^-$ (cf. Fig.~??). The subset $\mathscr{G}$ is
called the \defin{ergoregion}\index{ergoregion} and its boundary $\E:=\partial\mathscr{G}$
the \defin{ergosphere}. We shall discuss it further in connection with
the Penrose process in Chap.~\ref{s:ker}.

\section{Bifurcate Killing horizons} \label{s:sta:bifur_Killing_hor}



\section{The no-hair theorem}

In dimension $n=4$, one can go much further then just claiming that the
event horizon of a stationary black hole must be a Killing horizon.
One has indeed the \defin{Carter-Robinson theorem}\index{Carter-Robinson theorem}
(Carter 1971 \cite{Carte71}, Robinson 1975 \cite{Robin75}):
any stationary and axisymmetric 4-dimensional asymptotically flat
black hole spacetime $(\M,\w{g})$ that is
solution of the vacuum Einstein equation with a connected regular
event horizon $\Hor$ has a domain of outer communications that is isometric
to the domain of outer communications of the Kerr spacetime.

\begin{remark}
In their original works, Carter and Robinson assumed that $\Hor$ is a
\emph{non-degenerate}
Killing horizon, i.e. that the non-affinity coefficient $\kappa$ associated
with the Killing vector $\w{\chi}$ is non-zero. However this non-degeneracy
hypothesis can be released \cite{ChrusN10} (see \cite{ChrusLH12} for an
extended discussion).
\end{remark}

By combining the staticity, Israel, strong rigidity and Carter-Robinson theorems,
one arrives at the \defin{no-hair theorem}\index{no-hair theorem}:
\begin{greybox}
Any spacetime $(\M,\w{g})$ that
\begin{itemize}
\item is 4-dimensional
\item is asymptotically flat
\item is stationary
\item contains a black hole with a connected regular horizon
\item is analytic
\item is a solution of the vacuum Einstein equation
\end{itemize}
has a domain of outer communications that is isometric
to the domain of outer communications of the Kerr spacetime.
\end{greybox}


